\documentclass[11pt]{article}
\usepackage{listings}
%Umlaute
\usepackage[utf8]{inputenc}
 \usepackage[T1]{fontenc}
 %Ränder
 \usepackage[left=4cm,right=2cm,top=2cm,bottom=2cm]{geometry}
 %Grafiken
 \usepackage{graphicx}
 %Tiefe der Numerierung im document
 \setcounter{secnumdepth}{4}
 %Tiefe der Numerierung im Inhaltsverzeichnis
 \setcounter{tocdepth}{2}
 %Verlinken von verzeichnissen
 \usepackage{hyperref}
 %Farbänderung möglich
 \usepackage{color}
 %Farbe
 \definecolor{darkblue}{rgb}{0.3,0,0.5}
 %Links umfärben
 \hypersetup{colorlinks, linkcolor=darkblue}
 %Stichwordverzeichnis
 \usepackage{makeidx}
 \makeindex
 %Alles in Deutsch 
 \usepackage[german]{babel}
 \renewcommand{\indexname}{Stichwortverzeichnis}
\begin{document}
\begin{titlepage}
\title{Berühmte Sätze und Probleme Rund um Primzahlen}
\author{Ilja Kononenko und David Silvan Zingrebe}
\maketitle
Seminar: Diskrete Mathematik\\
Bei: Heinz-Juergen Schmidt\\
\end{titlepage}
\tableofcontents
\newpage
\section*{Vorwort}
Primzahlen sind ein wichtiger Bestandteil der Mathematik, wie wir sie heute kennen. Allerdings werden Primzahlen von den meißten Mathematikern als zufällig und unwichtig abgetan. Allein schon die Tatsache, daß man aus Primzahlen alle nicht primen Zahlen zusammensetzen kann, ist berauschend. Dieser Fakt wird im Kapitel über die Primfaktorzerlegung dieser Seminar genauer beschrieben. Ebenfalls taucht die Frage auf, ob es denn unendlich viele Primzahlen gibt. Dies wird im Kapitel über den Satz des Euklid abgearbeitet. Ist eine Zahl $n$ prim oder nicht? Das ist mit lediglich Stift und Papier nur sehr schwer herauszufinden und mit immer größer werdenden Zahlen stellt sich der Prozess als immer umständlicher und mühsamer heraus. Deshalb behandeln wir in dieser Seminararbeit auch zahlreiche Algorithmen, um herauszufinden, ob eine Zahl prim ist. Ebenfalls behandeln wir Algorithmen zum Finden ganzer Primzahlmengen in einem angegebenen Intervall. Doch Primzahlen sind nicht einfach nur Primzahlen. Es gibt verschiedene Arten von Primzahlen, die unterschiedlich zusammengesetzt sind. Mersenne Primzahlen sowie Fermatsche Primzahlen werden in dieser Seminararbeit genauer beschrieben. Wie in allen anderen Fachgebieten gibt es auch zu Primzahlen vielerlei Behauptungen, die bis heute noch nicht bewiesen worden sind. Dazu gehören die Primzahlzwillinge sowie die Goldbachsche Vermutung. In einigen Kapiteln sind Programmbeispiele zur Veranschaulichung der Themen beigelegt und beschrieben.

\newpage
% Grundlagen zu Primzahlen
\section{Definition und Grundlagen von Primzahlen}
Primzahlen sind Zahlen welche genau zwei Teiler haben. Sie sind deshalb ein sehr interessanter und wichtiger Bestandteil der Mathematik, obwohl sie nur im natürlichen Bereich der Zahlen eine große rolle Spielen.
			

\subsection{Primfaktorzerlegung}
\subsubsection{Definition}
Jede natürliche Zahl $n$ kann als Produkt von Primzahlen $p$, den sogenannten Primfaktoren, dargestellt werden. Diesen Vorgang nennt man Primfaktorzerlegung. Die Reihenfolge der einzelnen Primfaktoren $p$ spielt wie bei jeder anderen Multiplikation keine Rolle. Für den Fall, daß $n$ eine Primzahl ist, ist sie selbst ihr einziger Faktor. Bei der Primfaktorzerlegung können Faktoren mehrfach auftreten. Diese kann man Exponentiell zusammenfassen. Kanonische Primfaktorzerlegung wird eine Primfaktorzerlegung genannt, sobald die einzelnen Faktoren nach der Höhe ihrer Basis aufsteigend geordnet sind ($p_k < p_{k+1}$).
\subsubsection{Beispiele}
%*Beispiele für Primfaktorzerlegung*
$14 = 2 \cdot 7$\newline
$69 = 3 \cdot 23$\newline
$666 = 2 \cdot 3 \cdot 3 \cdot 27$\newline
$1337 = 7 \cdot 191$\newline\newline
Und in der Kanonischen Primfaktorzerlegung:\newline\newline
$666 = 2 \cdot 3^2 \cdot 27$
\subsubsection{Faktorisierungsverfahren}
Bis heute gibt es kein effizientes Faktorisierungsverfahren. Die einfachste Möglichkeit ist es, die zu faktorisierende, natürliche Zahl $n$ durch alle Primzahlen von Zwei bis zur Wurzel von $n$ zu teilen, bis man einen Teiler gefunden hat, bei dem das Ergebnis keinen Rest hat. Man merkt sich nun diesen Teiler, also den ersten Primfaktor, und führt nun fort, ersetzt jedoch $n$ mit dem gerade errechneten Quotienten. Fortgefürht wird dies, solange $n$ keine Primzahl ist. Ist sie es, hat man seine natürliche Zahl $n$ erfolgreich ausschließlich mit Primzahlen faktorisiert.
\subsubsection{Programmbeispiel}
Das folgende Programm in C++ generiert eine Abbildung als String der Primfaktorzerlegung der Zahlen von 2 bis g\_maxVal.
\lstset{language=C++}
\begin{lstlisting}[frame=single]
#include <vector>
#include <iostream>
#include <sstream>
#include <thread>
#include <map>
#include <fstream>

const size_t g_maxVal = 0xFFFFFF;

size_t teile(size_t zahl)
{
	size_t teiler = 2;
	size_t x = sqrt(zahl);
	while (zahl % teiler != 0)
	{
		if (teiler >= x)
		{
			return zahl;
		}
		else
		{
			++teiler;
		}
	}

	return teiler;
}

std::string primfaktorzerlegung(size_t zahl)
{
	std::stringstream ss;
	size_t teiler = 0;

	ss << zahl << " = ";

	while (zahl != 1)
	{
		teiler = teile(zahl);
		ss << teiler;
		zahl = zahl / teiler;
		if (zahl != 1)
			ss << "*";
	}

	return ss.str();
}

void generate_prime_factor(
	unsigned threadnum,
	unsigned threadc,
	std::map<size_t, std::string>* factors)
{
	if (threadnum == 0)
		threadnum += threadc;
	for (size_t i = threadnum; i < g_maxVal && !g_shouldExit;
		i += threadc)
	{
		factors->insert(
			std::pair<size_t, std::string>(
				i, primfaktorzerlegung(i)));
	}
}

void multithreaded_factorization(
	std::map<size_t, std::string>* factors)
{
	const size_t threadc = std::thread::hardware_concurrency();
	std::thread* threads = new std::thread[threadc];
	for (unsigned i = 0; i < threadc; ++i)
	{
		threads[i] = std::thread(
			generate_prime_factors,
			i, threadc, factors);
	}

	for (unsigned i = 0; i < threadc; ++i)
	{
		threads[i].join();
	}

	delete[] threads;
}

void print_factors(	std::ostream& os, 
			std::map<size_t, std::string>* factors)
{
	for (std::map<size_t, std::string>::iterator it =
			factors->begin();
			it != factors->end(); ++it)
	{
		os << it->second << "\n";
	}
}

int main(  )
{
	std::map<size_t, std::string> factors;
	multithreaded_factorization(&factors);
	std::ofstream os("testfile.txt",
		std::ios::trunc | std::ios::binary );
	if (os)
		print_factors(os, &factors);
	os.close();
	return 0;
}
\end{lstlisting}
\newpage
% Erklärung der einzelnen Methoden
In der main-Methode wird ein Array erstellt, in dem die Ergebnisse der Primfaktorzerlegung als String abgespeichert werden kann.
Anschließend wird die Methode \textbf{multithreaded\_factorization} aufgerufen, welche, um die Performance zu verbessern, mehrere Threads, deren Anzahl der Anzahl der Kerne der CPU entspricht, gestartet. Diese sollen die jeweiligen Faktorisierungen aller natürlichen Zahlen von 1 bis \textbf{g\_maxVal} berechnen und in \textbf{factors} abspeichern.\\
Die errechneten Faktoren können mittels der \textbf{print\_factors}-Methode in einen beliebigen stringstream \textbf{os} gestreamt werden. Dies können Dateien, Netzwerkübertragungen oder lediglich das Terminal des Computers sein.\\
Die Funktion \textbf{generate\_prime\_factors} durchläuft eine Schleife, welche die Primfaktorzerlegung der natürlichen Zahl \textbf{threadnum} berechnet und diesen Wert daraufhin um die Anzahl aller Threads \textbf{threadc} erhöht. Dadurch kommen sich die einzelnen Threads beim Abspeichern der errechneten Faktoren nicht in die Quere.\\
Die \textbf{primfaktorzerlegun}-Funktion gibt eine Stringrepräsentation der Faktorisierung der übergebenen natürlichen Zahl \textbf{zahl} zurück. Dabei wird solange \textbf{zahl} nicht eins, also der ersten Zahl, bei der eine weitere Faktorisierung keinen Sinn mehr macht, ist, der kleinste Teiler von \textbf{zahl} bestimmt, der eine Primzahl ist und an den zureckzugebenen String angehängt.\\
Der kleinstmögliche Teiler, der eine Primzahl ist, kann mit der Funktion \textbf{teile} bestimmt werden. Der gesuchte Teiler wird ermittelt, in dem geschaut wird, ob die übergebene Zahl dividiert mit dem Startwert 2 des Teilers \textbf{teiler} eine natürliche Zahl ergibt. Falls ja, wurde der kleinste Faktor gefunden. Falls nicht wird \textbf{teiler} inkrementiert und es wird erneut auf eine Division ohne Rest getestet. Ist \textbf{teiler} größer als die Quadratwurzel der gesuchten Zahl \textbf{zahl}, kann abgebrochen werden, da \textbf{zahl} prim ist und somit ihren bestmöglichen eigenen Faktor darstellt.\\

\newpage
\subsection{Der Satz des Euklid}
\newpage
\subsection{Simpler Primrechner}
Mit den Informationen, die wir bis jetzt erhalten haben, kann man bereits ein simples Programm schreiben, welches alle Primzahlen bis zu einer gegebenen natürlichen Zahl \textbf{Maxwert} findet.
\lstset{language=Java} 
\begin{lstlisting}[frame=single] 
Vector<Integer> g = new Vector<Integer>();
for (int i = 3; i< Maxwert; i++){
boolean isPrime = true;
			for (int j = 2; j < Math.sqrt(i); j++) {
				if (i%j==0) {
					isPrime = false;
					break;
				}
			}
		if(isPrime){
			g.add(i);
			}	
		}
\end{lstlisting}
Am Anfang muss man einen Behälter variabler Größe festlegen, da die Anzahl der Primzahlen bis \textbf{Maxwert} zu Beginn noch nicht bekannt ist. Anschließend wird eine for-Schleife durchlaufen, welche die zu testende Zahl gibt. Hier wird dann eine Boolsche Variable gesetzt, mit Hilfe welcher später abgefragt wird, ob dieser Test erfolgreich war oder nicht. Die Zahl i wird durch eine for-Schleife mit allen Zahlen von 2 bis zu ihrer Wurzel geteilt. Hat diese Division mit einer Zahl keinen Rest so ist klar, daß i nicht prim sein kann, die Boolsche Variable wird auf false gesetzt und aus der for-Schleife rausgebreakt. Hiermit wird verhindert, daß das Programm unnötig weiterprüft. Man muss dies nur bis zur Wurzel durchführen, da das Quadrieren der Wurzel der Zahl selbst gibt und somit wäre sie durch sich selbst teilbar und höher muss man nicht gehen da es laut der Primfaktorzerlegung einen kleineren Teiler bereits gegeben haben muss. Schlussendlich wird die Boolsche Variable abgefragt und wenn sie True ist wird die Primzahl in den Behälter geschrieben. Dieses Testverfahren wird mit jeder Zahl, bis zu einem vorbestimmten Maximum, durchgeführt. Der Benutzer könnte danach alle Primzahlen simple mit einer for-Schleife abfragen und sie wo immer er will ausgeben oder verwenden. Das Problem mit diesem Verfahren ist das es alle Zahlen durchgeht bis zu dem Maximalwert dadurch läuft es recht langsam.

\newpage
\subsection{Das Sieb des Eratosthenes}
Jetzt komme ich auf das Sieb des Eratosthenes zu sprechen welches ungefähr im Jahre 200 vor Christus vom namensgebenden Mathematiker Eratosthenes entwickelt wurde. Es ist genau wie der Simple Primrechner ein verfahren um eine große menge nach Primzahlen zu untersuchen und diese Herauszufiltern jedoch hat es einen großen Vorteil. Dieser ist das es effizienter ist. Dazu kann jeder dieses Prinzip verwenden wenn er genug Speicherkapazität hat, wobei mit dieser nicht nur die Elektrische Form gemeint ist sondern jede Art von Speicherung zum Beispiel ein Papier könnte als Speicher Fungieren. Es Funktioniert indem die Zahl '2' genommen wird und diese automatisch Prim ist daraufhin werden alle Multiplikationen, beginnend mit dem Quadrat, dieser Zahl gestrichen. Die nächste Zahl 'n' welche nicht gestrichen ist wird dann automatisch Prim und der Vorgang des Streichen wird mit dieser wiederholt. Dieser weitere Vorgang wird solange gemacht bis man zu der Stelle gelangt wo das Quadrat der aktuellen Zahl 'n' über dem Maximum liegt und dann werden alle noch nicht gestrichenen Zahlen in der liste automatisch Primzahlen.
\subsubsection{Programm}
Ich zeige ihnen dieses System jetzt als Veranschaulichung anhand des folgenden Java Programms:
\lstset{language=Java} 
\begin{lstlisting}[frame=single]
  private static boolean[] eratosthenes(int max)
	{
		boolean[] primes = new boolean[max];
		for (int i = 2; i < max; ++i)
		{
			primes[i] = true;
		}
		
		int i = 2;
		for(; i * i < max; ++i)
		{
			if (primes[i])
			{
				//System.out.println(i);
				for (int j = i * i; j < max; j = j + i)
					primes[j] = false;
			}
		}
		
		
		return primes;
	}
\end{lstlisting}
\newpage
\subsubsection{Programmerklärung}
Das Programm erstellt ein array aus boolwerten welches so groß ist wie ein vorreingetragener Maximalwert, wobei die Indizes für die Primzahlen stehen.Da 1 und 0 keine Primzahlen sind werden beginnend mit dem Index 2 als true initialisiert damit alle Indizes für Testzwecke zur verfügung stehen. Anschließend beginnt das sieb mit der Zahl '2' und wenn diese true ist werden alle Multiplikationen von '2' gestrichen. Im nächsten schritt werden alle gestrichenen Zahlen übersprungen indem abgefragt wird ob sie true sind. Wenn die nächste gefunden wurde wird beginnend mit dem Quadratt der Zahl alle Multiplikationen gestrichen. Dieser Vorgang wird wiederholt bis das Quadratt einer Zahl größer ist als das Maximum. Somit hat man jetzt ein Array wo nur die Primzahlen true sind und man kann diese durch eine Abfrage simpel ausgeben.
\newpage
%spezielle Primzahlen
\section{Spezielle Primzahlen}
In diesem Abschnitt werden sie über zwei der berühmtesten Primzahl arten aufgeklärt zuerst über die Mersenne Primzahlen und anschließend über die Fermatschen Primzahlen. Diese sind extrem wichtig da sie die Rekorde Halten als aktuell größte gefundene Primzahlen.
\subsection{Mersenne Primzahlen}
Die Mersenne Zahlen werden im allgemeine als $2^{n}-1 $ definiert, aber nicht jede von ihnen ist Prim. Sie sind so berühmt und geschätzt da man sie leicht berechnen kann und es einfacher ist zu überprüfen ob sie Prim sind oder nicht. Dazu da sie eine Wachstumsfunktion als Definition haben werden sie sehr schnell zu großen Zahlen.
\subsubsection{Geschichte der Mersennen Zahlen}
Die Mersenne Zahlen sind schon seit mehreren Jahrtausenden bekannt und die man glaubte das alle Zahlen der Form $2^{n}-1$ Prim sind. Die erste Zahl welche in dieser Art ist und nicht Prim ist, ist 2047. Dies bewies Hudalcrius Regius im Jahre 1536. Anschließend stellte Pietro Cataldi 1603 die erste Aussage auf mit den Hochzahlen 17,19,31 und 37, welche aber von Pierre Fermat im Jahre 1640 korriegiert wurde das 23 und 37 nicht Prim sind. Der Namensgeber Marin Mersenne fertigte die erste größere liste an mit: 
\[n=2,3,5,7,13,17,19,31,67,127\; und\; 257\]
in seinem Vorwort zu seinem Cognita Physica-Mathematica im Jahre 1644.
Obwohl sie aus heutiger Sicht inkorrekt ist wurde sein Name der Name dieser Primzahlen. Es dauerte mehrere Jahrhunderte bis alle zahlen überprüft wurden und erst 300 Jahre später im Jahr 1947 war sie dann vollständig bis $n<258$ überprüft worden. Durch die fortschrittlichen Entwicklungen steigerten sich die Mersenne Primzahlen schnell in den darauffolgenden Jahren. Heutzutage kann jeder die Mersenne Primzahlenfindung unterstützen indem er seine Rechnerleistung für die Great Internet Mersenne Prime Search besser bekannt als GIMPS aufgibt.
\newpage
\subsection{Fermatsche Primzahlen}
\newpage
% mathematische Vermutungen zu Primzahlen
\section{Mathematische Vermutungen zu Primzahlen}
Hier kommen wir zu den Unbelegten Vermutungen von Primzahlen. Wir werden versuchen die Probleme mit diesem Vermutungen zu schildern und erklären warum sie nicht belegt sind und welche Probleme auftreten wenn man sie zu belegen versucht.
\subsection{Primzahl-Zwillinge}
\newpage
\subsection{Goldbachsche Vermutung}
\newpage
\section*{Info für sie nicht in die Bewertung mit einziehen}
Welche Sachen von wem bearbeitet werden/wurden:
Vorwort:Ilja\\
Infos zu den Sections:Ilja\\
Primfakzorzerlegung: Silvan\\
Der Satz des Euklid: Silvan\\
Simpler Primrechner: Ilja\\
Das Sieb des Eratosthenes:Ilja\\
Mersenne Primzahlen: Ilja\\
Fermatsche Primzahlen: Silvan\\
Primzahl Zwillinge: Ilja\\
Goldbachsche Vermutung: Silvan\\
Dazu möge gesagt sein falls sie sich wundern wieso es keine bzw kaum quellen gibt wir hatten bereits ein großes Vorwissen über Primzahlen und mussten uns daher keinen Auszählbaren Quellen bedienen.

\end{document}