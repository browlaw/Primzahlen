\subsubsection{Definition}
Der Satz des Euklid besagt, daß es unendlich viele Primzahlen gibt.
\subsubsection{Geschichte}
Wie man dem Namen entnehmen kann, wurde dieser Satz vom griechischem Mathematiker Euklid, der im 3. Jahrhundert nach Christus lebte, aufgestellt.
\subsubsection{Beweis des Satz des Euklid}
Geht man von einer endlichen Menge Primzahlen $p_0$ bis $p_n$ aus, deren Produkt $v$ ist, gibt es zwei mögliche Ereignisse für $v + 1$. Das Ergebnis kann prim oder nicht prim sein. Ist $v + 1$ prim, so ist bereits bewiesen, daß die bisher bekannten Primzahlen nicht die einzigen sind und es noch weitere gibt. Ist $v + 1$ hingegen nicht prim, so kann man dessen Primfaktorzerlegung bilden. Diese Primfaktoren $p_0$ bis $p_n$ enthalten nun einen Faktor $q$, welcher sowohl von $v$, als auch von $v + 1$ ein Teiler ist. Die einzige Möglichkeit für $q$ ist also $1$, was jedoch keine Primzahl ist. Demnach ist die Aussage, daß es endlich viele Primzahlen gibt, widersprüchlich und falsch. Man kann schließen, daß es unendlich viele Primzahlen gibt.

\subsubsection{Anwendung}
Der Satz des Euklid findet Anwendung bei der Findung weiterer Primzahlen. Kennt man bereits eine oder mehrere Primzahlen, multipliziert man diese miteinander und addiert $1$ auf das erhaltene Produkt und es gibt zwei Möglichkeiten die auftreten können. Entweder ist das Ergebnis prim und nicht in der Menge der bereits bekannten  Primzahlen enhalten, da sie mindestens $1$ größer ist, oder man bildet die Primfaktorzerlegung des Ergebnisses. Mindestens einer dieser Faktoren ist eine bisher noch unbekannt Primzahl.
% Beispiele für das Finden neuer Primzahlen
\subsubsection{Beispiele}
Man besitze eine endliche Menge an Primzahlen $M_p\{3; 5; 11\}$, multipliziert diese miteinander und addiert eins.\newline
$3 \cdot 5 \cdot 11 + 1 = 166$\newline
Von dem Ergebnis bildet man nun die Primfaktorzerlegung und erhält neue Primzahlen.\newline
$166 = 2 \cdot 83$\newline
Die neuen Primzahlen sind 2 sowie 83.\newline
Besitzt man beispielsweise jedoch nur die Primzahl $3$, so addiert man lediglich 1 und wendet wieder die Primfaktorzerlegung an. $3 + 1 = 4 \Rightarrow 4 = 2^2$. Somit erhlält man die zuvor noch unbekannt Primzahl 2.
