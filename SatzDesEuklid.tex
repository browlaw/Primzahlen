\subsubsection{Definition}
Der \textbf{Satz des Euklid} besagt, dass es unendlich viele Primzahlen gibt.
\subsubsection{Geschichte}
Wie man dem Namen entnehmen kann, wurde dieser Satz vom griechischem Mathematiker \textbf{Euklid}, der im 3. Jahrhundert nach Christus lebte, aufgestellt.
*weitere Infos*
\subsubsection{Beweis des Satz des Euklid}
Geht man von einer endlichen Menge Primzahlen $p_0$ bis $p_n$ aus, deren Produkt $v$ ist gibt es zwei moegliche Ereignisse fuer $v + 1$.
Ist $v + 1$ prim, so wurde belegt, dass  $p_0$ bis $p_n$ nicht die einzigen Primzahlen sind.
Ist $v + 1$ nicht prim, so enthaelt $p_0$ bis $p_n$ einen Primfaktor $q$, welcher sowohl von $v$ als auch von $v + 1$ ein Teiler ist. Da die einzige Moeglichkeit fuer $q$ $1$ waere, was keine Primzahl ist, ist dies nicht moeglich: ein Widerspruch.
Durch diesen Widerspruchsbeweis wird also bewiesen, dass die Anzahl an Primzahlen nicht endlich ist.
\subsubsection{Anwendung}
Errechnung weiterer Primzahlen mit Hilfe der Primfaktorzerlegung...