Die Perfekten Primzahl Zwillinge existieren nur einmal und zwar sind das 2 und 3. Alle darauffolgenden Primzahl Zwillinge sind definiert mit einem Abstand von genau zwei, da alle geraden zahlen durch zwei teilbar sind. 
\subsubsection{Eigenschaften von Primzahl-Zwillingen}
Die Haupteigenschaft aller Primzahl Zwillinge außer 3 und 5 ist das sie um eine durch 6 teilbare zahl gehen weshalb sie auch als eine Form von 
$6n-1\;und\;6n+1$ definiert werden können. Dies hat denn Grund da jede mit fünf endende zahl keine Primzahl ist. Es ist jedoch nicht gewiss ob es unendlich viele Primzahl Zwillinge gibt obwohl es unendlich viele Primzahlen gibt, nach dem Satz des Euklid.
\subsubsection{Primzahl-Zwillings Vermutung}
Die Primzahl-Zwillings Vermutung besagt das es unendlich viele Primzahlen $p$ gibt sodass $p+2$ auch prim ist. Dazu gibt es eine verallgemeinerte Version von Alphonse de Polignac das für jede natürliche zahl $n$ auch gilt, es gibt unendlich viele Primzahlen $p$ sodass $p+2k$ auch prim ist, bei $k=1$ ist es dann die Primzahl-Zwillings Vermutung. Da die Wahrscheinlichkeit das eine zahl $n$ Prim ist liegt bei $1/log(n)$ somit könnte man vermuten das $n+2$ auch prim ist wenn es als eigenständige zahl sieht $\pi_2(n)\sim\frac{n}{\log^2(n)}$.
\subsubsection{Erste Hardy-Littlewood Vermutung}
Die Primzahl-Zwillings Konstante $C_2$ ist definiert als $C_2=\prod\limits_{p\geq3}(\frac{p(p-2)}{(p-1)^2})\approx0.66016181584...$.
Dazu kann man nach Godfray Harold Hardy und John Edensor Littlewood vermuten das $\pi_2(n)\sim2C_2\frac{n}{\log^2(n)}$.
Dieser Vermutung kann zwar zugestimmt werden, aber sie gilt noch als unbewiesen.