Die perfekten Primzahlzwillinge existieren nur einmal, nämlich die $2$ und die $3$. Alle darauffolgenden Primzahlzwillinge sind mit einem Abstand von genau 2 zueinander definiert, da alle geraden Zahlen durch zwei teilbar sind. 
\subsubsection{Eigenschaften von Primzahl-Zwillingen}
Die Haupteigenschaft aller Primzahlzwillinge außer 3 und 5 ist, daß sie um eine durch 6 teilbare Zahl gehen, weshalb sie auch als eine Form von $6n-1\;und\;6n+1$ definiert werden können. Dies hat den Grund, daß jede auf Fünf endende Zahl keine Primzahl ist. Es ist jedoch nicht gewiss, ob es unendlich viele Primzahlzwillinge gibt, obwohl es nach Euklid unendlich viele Primzahlen gibt.
\subsubsection{Primzahl-Zwillings Vermutung}
Die Primzahl-Zwillings Vermutung besagt, daß es unendlich viele Primzahlen $p$ gibt, sodaß $p+2$ auch prim ist. Dazu gibt es eine verallgemeinerte Version von Alphonse de Polignac, daß für jede natürliche Zahl $n$ auch gilt, es gibt unendlich viele Primzahlen $p$ sodass $p+2k$ auch prim ist, bei $k=1$ ist es dann die Primzahl-Zwillings Vermutung. Da die Wahrscheinlichkeit das eine zahl $n$ Prim ist liegt bei $1/log(n)$ somit könnte man vermuten das $n+2$ auch prim ist wenn es als eigenständige zahl sieht $\pi_2(n)\sim\frac{n}{\log^2(n)}$.
\subsubsection{Erste Hardy-Littlewood Vermutung}
Die Primzahl-Zwillings Konstante $C_2$ ist als definiert $C_2=\prod\limits_{p\geq3}(\frac{p(p-2)}{(p-1)^2})\approx0.66016181584...$.
Dazu kann man nach Godfray Harold Hardy und John Edensor Littlewood vermuten, daß $\pi_2(n)\sim2C_2\frac{n}{\log^2(n)}$.
Dieser Vermutung kann zwar zugestimmt werden, aber sie gilt noch als unbewiesen.
