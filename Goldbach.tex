\subsection{Definition}
Christian Goldbach hat als wahrscheinlich erster vermutet, dass jede gerade Zahl, die groesser als 2 ist, als Summe von zwei Primzahlen dargestellt werden kann. Dies ist die Starke Goldbachsche Vermutung. Aus dieser Starken Vermutung geht die Schwache Goldbachsche Vermutung hervor, welchehingegen aussagt, dass jede ungerade Zahl, welche groesser als 5 ist, als Summe von 3 Primzahlen dargestellt werden kann. Bis heute gibt es keinen Beweis fuer die Goldbachsche Vermutung, verschiedene Computerberechnungen zeigen jedoch, dass die Vermutung  bis Zahlengroessen bis zu $4\cdot10^18$ zutrifft. Auch die Moeglichkeiten, gerade Zahlen groesser als 2 als Summe zweier Primzahlen darzustellen, steigt, je groesser die gerade Zahl ist. Es laesst sich also vermuten, dass die Vermutung wahr ist, bewiesen ist sie dennoch nicht.
