\subsection{Definition}
Christian Goldbach hat als wahrscheinlich erster vermutet, dass jede gerade Zahl, die groesser als 2 ist, als Summe von zwei Primzahlen dargestellt werden kann. Dies ist die Starke Goldbachsche Vermutung. Aus dieser Starken Vermutung geht die Schwache Goldbachsche Vermutung hervor, welchehingegen aussagt, dass jede ungerade Zahl, welche groesser als 5 ist, als Summe von 3 Primzahlen dargestellt werden kann. Bis heute gibt es keinen Beweis fuer die Goldbachsche Vermutung, verschiedene Computerberechnungen zeigen jedoch, dass die Vermutung  bis Zahlengroessen bis zu $4\cdot10^{18}$ zutrifft. Auch die Moeglichkeiten, gerade Zahlen groesser als 2 als Summe zweier Primzahlen darzustellen, steigt, je groesser die gerade Zahl ist. Es laesst sich also vermuten, dass die Vermutung wahr ist, bewiesen ist sie dadurch jedoch nicht. Der deutsche Mathemathiker David Hilbert nach die Starke Goldbachsche Vermutung als sein achtes Problem in seine Liste "Hilbertsche Probleme" auf.
\subsection{Starke Goldbachsche Vermutung}
Die Starke Goldbachsche Vermutung, oder auch Binaere Goldbachsche Vermutung genannt, besagt, dass jede gerade Zahl $n$, die groesser als 2 ist, als Summe zweier Primzahlen abgebildet werden kann. Also $n = p + q$, wobei $p$ und $q$ die beiden Primzahlen sind. Generell ist die Darstellung von $n$ nicht eindeutig. $n$ kann in den meisten Faellen als Summe verschiedener Primzahlpaare dargestellt werden.
\subsection{Schwache Goldbachsche Vermutung}
Die Schwache Goldbachsche Vermutung, oder Ternaere Goldbachsche Vermutung, geht aus der Starken Goldbachschen Vermutung hervor. Sie besagt, dass ungerade Zahlen $u$, die groesser als $5$ sind, als Summe dreier Primzahlen dargestellt werden kann: $u = p + q + r$. Ist der Umstand gegeben, dass die Starke Goldbachsche Vermutung wahr ist, so ist $u = (u - 3) + 3$. $u - 3$ ist hierbei nach Goldbach die Summe zweier Primzahlen, also $u - 3 = p + q$. Somit ist die Darstellung von $u$ als Summe von 3 Primzahlen gegeben: $u = p + q + 3$. Wie bei der Starken Vermutung ist die Abbildung nicht eindeutig. $u$ kann also, mit Ausnahme von der Primzahl $7$, durch unterschiedliche Summanden abgebildet werden.
