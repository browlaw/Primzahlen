\subsubsection{Definition}
Fermatsche Primzahlen sind Fermat-Zahlen, die prim sind. Fermat-Zahlen sind hierbei Zahlen der Form $F_n = 2^{2^n} + 1$, wobei $n \ge 0$.
\subsubsection{Informationen}
Pierre de Fermat, nach dem die Fermatschen Zahlen benannt sind, behauptete, daß alle Fermatschen Zahlen prim seien. Allerdings fand Leonhard Euler im Jahre 1732 mit der $641$ einen Teiler von $F_5=4294967297$. Bis heute wurde keine Fermatsche Primzahl mit $n > 5$ gefunden und es wird vermutet, daß es nach den ersten fünf Fermatschen Zahlen keine weiteren Fermatschen Zahlen gibt, die prim sind. Diese Aussage beruht auf dem Primzahlsatz, nach dem die Anzahl der Primzahlen kleiner oder gleich $n$ circa $n / \ln n$ ist. Demnach ist die Wahrscheinlichkeit dafür, daß $F_n$ eine Primzahl ist, etwa $2 / \ln_{F_n}$. Die Wahrscheinlichkeit dafür, daß $F_6$ prim ist, beträgt also $\frac{2}{\ln_{F_6}} = 4.508422\%$ und für $F_7$ bereits nur noch $\frac{2}{\ln_{F_7}} = 2.254211\%$. Erkennbar ist hierbei, daß sich die Wahrscheinlichkeit für das Prim-Sein von $p_{F_{n+1}}$, immer circa halb so groß ist, wie $p_{F_{n}}$. Dies hängt damit zusammen, daß die Fermatsche Zahl $F_{n+1}$ immer etwa doppelt so viele Dezimalstellen wie $F_{n}$ besitzt.
\subsubsection{Rekursive Berechnung von Fermatschen Zahlen}
Fermatsche Zahlen können durch Rekursion berechnet werden. Hierfür gibt es verschiedene Algorithmen.
Für $n \ge 1$ gilt:
\begin{equation}
F_n = (F_{n - 1} - 1 )^2 + 1
\end{equation}
oder
\begin{equation}
F_n = F_0 \cdot F_1 \cdot ... \cdot F_{n-1} + 2
\end{equation}
Hier ist der erste dieser beiden Algorithmen in Programmcode umgesetzt:
\lstset{language=C}
\begin{lstlisting}[frame=single]
#include <stdio.h>
#include <stdlib.h>
#include <math.h>

double read_input(void)
{
        char buf[256];
        double n = -1;

        printf("Enter n: ");
        scanf("%lf", &n);
        return n;
}

double fermat_1(double num)
{
        double f;
        if (num == 0)
                f = 3;
        else
                f = pow(fermat_1(num - 1) - 1, 2) + 1;
        printf("F_%.0lf = %.0lf\n", num, f);
        return f;
}

int main(int argc, char *argv[])
{
        double n = read_input();
        fermat_1(n);

        return 0;
}
\end{lstlisting}
\subsubsection{Anwendung Fermatscher Primzahlen in der Geometrie}
Carl Friedrich Gauß fand einen Zusammenhang zwischen den Fermatschen Primzahlen und der Konstruktion von regelmäßigen Vielecken. Er bewies, daß ein regelmäßiges Vieleck mit $n$ Seiten nur mit ausschließlich Zirkel und Lineal konstruiert werden kann, wenn $n$ eine Potenz von $2$ oder das Produkt einer Protenz von 2 und verschiedenen Fermatschen Primzahlen ist.
\subsubsection{Finden neuer Primzahlen}
Fermatschen Zahlen sind teilerfremd zueinander. Daraus kann man schließen, daß die Faktoren einer Primfaktorzerlegung sich nicht überschneiden. Da es unendlich viele Fermatsche Zahlen gibt, kann man erschließen, daß es unendlich viele unterschiedliche Primfaktoren und somit unendlich viele Primzahlen gibt.
