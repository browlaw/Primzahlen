Jede natuerliche Zahl 'n' kann als Produkt von Primzahlen 'p' dargestellt werden. Dies nennt man Primfaktorzerlegung. Die Reihenfolge der einzelnen Primfaktoren 'p' spielt wie bei jeder anderen Multiplikation keine Rolle. Fuer den Fall, dass 'n' eine Primzahl ist, ist sie ihr einziger Faktor. Bei der Primfaktorzerlegung koennen Faktoren mehrfach auftreten. Diese kann man Exponentiell zusammenfassen. Kanonische Primfaktorzerlegung wird eine Primfaktorzerlegung genannt, sobald die einzelnen Faktoren aufsteigend geordnet sind ('pk < pk+1').

*Beispiele fuer Primfaktorzerlegung*

Verfahren: Bis jetzt kein effizienter Algorithmus