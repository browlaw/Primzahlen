\subsubsection{Definition}
Jede natuerliche Zahl $n$ kann als Produkt von Primzahlen $p$, den Primfaktoren, dargestellt werden. Dies nennt man Primfaktorzerlegung. Die Reihenfolge der einzelnen Primfaktoren $p$ spielt wie bei jeder anderen Multiplikation keine Rolle. Fuer den Fall, dass $n$ eine Primzahl ist, ist sie ihr einziger Faktor. Bei der Primfaktorzerlegung koennen Faktoren mehrfach auftreten. Diese kann man Exponentiell zusammenfassen. Kanonische Primfaktorzerlegung wird eine Primfaktorzerlegung genannt, sobald die einzelnen Faktoren nach der Hoehe ihrer Basis aufsteigend geordnet sind ($p_k < p_{k+1}$).
\subsubsection{Beispiele}
%*Beispiele fuer Primfaktorzerlegung*
$14 = 2 * 7$\newline
$69 = 3 * 23$\newline
$666 = 2 * 3 * 3 * 27$\newline
$1337 = 7 * 191$\newline\newline
Und in der Kanonischen Primfaktorzerlegung:\newline\newline
$666 = 2 * 3^2 * 27$
\subsubsection{Faktorisierungsverfahren}
Bis heute gibt es kein effizientes Faktorisierungsverfahren. Die einfachste Moeglichkeit ist es, die zu faktorisierende, natuerliche Zahl \textbf{n} durch alle Primzahlen von Zwei bis zur Wurzel von \textbf{n} zu teilen, bis man einen Teiler gefunden hat, bei dem das Ergebnis keinen Rest hat. Man merkt sich nun diesen Teiler, also den ersten Primfaktor, und fuehrt nun fort, ersetzt jedoch \textbf{n} mit dem gerade errechneten Quotienten. Fortgefuerht wird dies, solange \textbf{n} keine Primzahl ist. Ist sie es, hat man seine natuerliche Zahl \textbf{n} erfolgreich ausschliesslich mit Primzahlen faktorisiert.
\subsubsection{Programmbeispiel}
Das folgende Programm in C++ generiert eine Abbildung als String der Primfaktorzerlegung der Zahlen von 2 bis g\_maxVal.
\lstset{language=C++}
\begin{lstlisting}[frame=single]
#include <vector>
#include <iostream>
#include <sstream>
#include <thread>
#include <map>
#include <fstream>

const size_t g_maxVal = 0xFFFFFF;

size_t teile(size_t zahl)
{
	size_t teiler = 2;
	size_t x = sqrt(zahl);
	while (zahl % teiler != 0)
	{
		if (teiler >= x)
		{
			return zahl;
		}
		else
		{
			++teiler;
		}
	}

	return teiler;
}

std::string primfaktorzerlegung(size_t zahl)
{
	std::stringstream ss;
	size_t teiler = 0;

	ss << zahl << " = ";

	while (zahl != 1)
	{
		teiler = teile(zahl);
		ss << teiler;
		zahl = zahl / teiler;
		if (zahl != 1)
			ss << "*";
	}

	return ss.str();
}

void generate_prime_factor(
	unsigned threadnum,
	unsigned threadc,
	std::map<size_t, std::string>* factors)
{
	if (threadnum == 0)
		threadnum += threadc;
	for (size_t i = threadnum; i < g_maxVal && !g_shouldExit;
		i += threadc)
	{
		factors->insert(
			std::pair<size_t, std::string>(
				i, primfaktorzerlegung(i)));
	}
}

void multithreaded_factorization(
	std::map<size_t, std::string>* factors)
{
	const size_t threadc = std::thread::hardware_concurrency();
	std::thread* threads = new std::thread[threadc];
	for (unsigned i = 0; i < threadc; ++i)
	{
		threads[i] = std::thread(
			generate_prime_factors,
			i, threadc, factors);
	}

	for (unsigned i = 0; i < threadc; ++i)
	{
		threads[i].join();
	}

	delete[] threads;
}

void print_factors(	std::ostream& os, 
			std::map<size_t, std::string>* factors)
{
	for (std::map<size_t, std::string>::iterator it =
			factors->begin();
			it != factors->end(); ++it)
	{
		os << it->second << "\n";
	}
}

int main(  )
{
	std::map<size_t, std::string> factors;
	multithreaded_factorization(&factors);
	std::ofstream os("testfile.txt",
		std::ios::trunc | std::ios::binary );
	if (os)
		print_factors(os, &factors);
	os.close();

	std::system("pause");

    return 0;
}
\end{lstlisting}

In der main-Methode wird ein Array erstellt, in dem die Ergebnisse der Primfaktorzerlegung als String abgespeichert werden kann. Anschliessend wird die Methode \textbf{multithreaded\_factorization} aufgerufen, welche, um die Performance zu verbessern, mehrere Threads, deren Anzahl der Anzahl der Kerne der CPU entspricht, gestartet. Diese sollen die jeweiligen Faktorisierungen aller natuerlichen Zahlen von 1 bis \textbf{g\_maxVal} berechnen und in \textbf{factors} abspeichern. Die errechneten Faktoren koennen mittels der \textbf{print\_factors}-Methode in einen beliebigen stringstream \textbf{os} gestreamt werden. Dies koennen Dateien, Netzwerkuebertragungen oder lediglich das Terminal des Computers sein. Die Funktion \textbf{generate\_prime\_factors} durchlaeuft eine Schleife, welche die Primfaktorzerlegung der natuerlichen Zahl \textbf{threadnum} berechnet und diesen Wert daraufhin um die Anzahl aller Threads \textbf{threadc} erhoeht. Dadurch kommen sich die einzelnen Threads beim Abspeichern der errechneten Faktoren nicht in die Quere. Die \textbf{primfaktorzerlegun}-Funktion gibt eine Stringrepraesentation der Faktorisierung der uebergebenen natuerlichen Zahl \textbf{zahl} zurueck. Dabei wird solange \textbf{zahl} nicht eins, also der ersten Zahl, bei der eine weitere Faktorisierung keinen Sinn mehr macht, ist, der kleinste Teiler von \textbf{zahl} bestimmt, der eine Primzahl ist und an den zurueckzugebenen String angehaengt. Der kleinstmoegliche Teiler, der eine Primzahl ist, kann mit der Funktion \textbf{teile} bestimmt werden. Der gesuchte Teiler wird ermittelt, in dem geschaut wird, ob die uebergebene Zahl dividiert mit dem Startwert 2 des Teilers \textbf{teiler} eine natuerliche Zahl ergibt. Falls ja, wurde der kleinste Faktor gefunden. Falls nicht wird \textbf{teiler} inkrementiert und es wird erneut auf eine Division ohne Rest getestet. Ist \textbf{teiler} groesser als die Quadratwurzel der gesuchten Zahl \textbf{zahl}, kann abgebrochen werden, da \textbf{zahl} prim ist und somit ihren bestmoeglichen eigenen Faktor darstellt.