In dieser Nachfolgenden Seminararbeit werden sie über Primzahlen informiert mithilfe von
berühmten Sätzen und Problemen rund um Primzahlen. Zuallererst informieren wir sie über die allgemeinen Informationen um Primzahlen wo auch schon die frage aufkommt ob es unendlich viele davon gibt diese klärt der Satz des Euklid auf. Aber es gibt noch die Frage was die Restlichen Natürlichen Zahlen sind? Mit dieser Frage beschäftigt sich die Primfaktorzerlegung. Woraufhin wir auf das nächste Problem stoßen und zwar wie erkennt man möglichst schnell welche zahlen Prim sind und welche nicht? Eins der ältesten und das aktuell wahrscheinlich effizienteste Verfahren dafür ist das Sieb des Eratosthenes.
Danach kommen wir auf die Speziellen Primzahlen zu sprechen welche die Mersenne und die Fermatsche Primzahlen sind und wie diese Zustande kommen. Zuallerletzt beschäftigen wir uns mit Mathematischen Vermutungen zu Primzahlen welche noch unbewiesen sind. Da kommen wir zuerst zu den Primzahl Zwillingen zu sprechen, welche simpel gesagt Zwei Primzahlen nebeneinander sind. Und Anschließend zur Goldbachschen Vermutung welche sich in die Starke und die Schwache Goldbachsche Vermutung aufteilt. Zu vielen der oben genannten Themen sollten sie auch noch ein oder mehrere Programme, welche diese erklären oder verwenden, finden. 