Primzahlen sind ein wichtiger Bestandteil der Mathematik, wie wir sie heute kennen. Allerdings werden Primzahlen von den meißten Mathematikern als zufällig und unwichtig abgetan. Allein schon die Tatsache, daß man aus Primzahlen alle nicht primen Zahlen zusammensetzen kann, ist berauschend. Dieser Fakt wird im Kapitel über die Primfaktorzerlegung dieser Seminar genauer beschrieben. Ebenfalls taucht die Frage auf, ob es denn unendlich viele Primzahlen gibt. Dies wird im Kapitel über den Satz des Euklid abgearbeitet. Ist eine Zahl $n$ prim oder nicht? Das ist mit lediglich Stift und Papier nur sehr schwer herauszufinden und mit immer größer werdenden Zahlen stellt sich der Prozess als immer umständlicher und mühsamer heraus. Deshalb behandeln wir in dieser Seminararbeit auch zahlreiche Algorithmen, um herauszufinden, ob eine Zahl prim ist. Ebenfalls behandeln wir Algorithmen zum Finden ganzer Primzahlmengen in einem angegebenen Intervall. Doch Primzahlen sind nicht einfach nur Primzahlen. Es gibt verschiedene Arten von Primzahlen, die unterschiedlich zusammengesetzt sind. Mersenne Primzahlen sowie Fermatsche Primzahlen werden in dieser Seminararbeit genauer beschrieben. Wie in allen anderen Fachgebieten gibt es auch zu Primzahlen vielerlei Behauptungen, die bis heute noch nicht bewiesen worden sind. Dazu gehören die Primzahlzwillinge sowie die Goldbachsche Vermutung. In einigen Kapiteln sind Programmbeispiele zur Veranschaulichung der Themen beigelegt und beschrieben.
